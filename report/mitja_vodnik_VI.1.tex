\documentclass[a4paper]{article}
\usepackage[slovene]{babel}
\usepackage[utf8]{inputenc}
\usepackage[T1]{fontenc}
%\usepackage[margin=2cm, bottom=3cm, foot=1.5cm]{geometry}
\usepackage{float}
\usepackage{graphicx}
\usepackage{amsmath}
\usepackage{amssymb}
\usepackage{subcaption}
\usepackage{hyperref}
\usepackage{dirtytalk}

\newcommand{\tht}{\theta}
\newcommand{\Tht}{\Theta}
\newcommand{\dlt}{\delta}
\newcommand{\eps}{\epsilon}
\newcommand{\thalf}{\frac{3}{2}}
\newcommand{\ddx}[1]{\frac{d^2#1}{dx^2}}
\newcommand{\ddr}[2]{\frac{\partial^2#1}{\partial#2^2}}
\newcommand{\mddr}[3]{\frac{\partial^2#1}{\partial#2\partial#3}}

\newcommand{\der}[2]{\frac{d#1}{d#2}}
\newcommand{\pder}[2]{\frac{\partial#1}{\partial#2}}
\newcommand{\half}{\frac{1}{2}}
\newcommand{\forth}{\frac{1}{4}}
\newcommand{\q}{\underline{q}}
\newcommand{\p}{\underline{p}}
\newcommand{\x}{\underline{x}}
\newcommand{\liu}{\hat{\mathcal{L}}}
\newcommand{\bigO}[1]{\mathcal{O}\left( #1 \right)}
\newcommand{\pauli}{\mathbf{\sigma}}
\newcommand{\bra}[1]{\langle#1|}
\newcommand{\ket}[1]{|#1\rangle}
\newcommand{\id}[1]{\mathbf{1}_{2^{#1}}}
\newcommand{\tinv}{\frac{1}{\tau}}
\newcommand{\s}{\sigma}
\newcommand{\vs}{\vec{\s}}
\newcommand{\vr}{\vec{r}}
\newcommand{\vq}{\vec{q}}
\newcommand{\vv}{\vec{v}}
\newcommand{\vo}{\vec{\omega}}
\newcommand{\uvs}{\underline{\vs}}
\newcommand{\expected}[1]{\left\langle #1 \right\rangle}
\newcommand{\D}{\Delta}

\newcommand{\range}[2]{#1, \ldots, #2}
\newcommand{\seq}[2]{#1 \ldots #2}
\newcommand{\psiCoef}[2]{\psi_{\range{#1}{#2}}}
\newcommand{\psiCoeff}[3]{\psi_{#1, \range{#2}{#3}}}
\newcommand{\mpa}[2]{#1^{(#2)}_{s_#2}}
\newcommand{\us}{\underline{s}}
\newcommand{\up}{\uparrow}
\newcommand{\down}{\downarrow}

\begin{document}

    \title{\sc\large Višje računske metode\\
		\bigskip
		\bf\Large Problem elektronske strukture: metode povprečnega polja in DFT}
	\author{Mitja Vodnik, 28182041}
            \date{\today}
	\maketitle

    Kot preprost zgled uporabe DFT metode računamo elektronsko strukturo helijevega atoma v osnovnem stanju.
    Zanima nas torej elektronska gostota $\rho(\vr)$, ki po Hohenberg-Kohnovem izreku enolično določa osnovno
    stanje mnogodelčnega kvantno-mehanskega sistema.
    Ker ima helijev atom dva elektrona, je gostoto mogoče izraziti z le eno sferno simetrično orbitalo:

    \begin{equation}\label{eq1}
        \rho(\vr) = \rho(r) = 2|\phi(r)|^2
    \end{equation}

    To je orbitala, ki jo uporabimo v Kohn-Shamovem modelu za funkcional elektroske gostote.
    Tako dobimo DFT metodo, ki je ekvivalentna reševanju naslednjih Hartree-Fockovih enačb za orbitalo:

    \begin{equation}\label{eq2}
        \left( -\half \nabla^2 + V_{eff}(\vr) \right) \phi(\vr) = \epsilon \phi(\vr)
    \end{equation}

    \begin{equation}\label{eq3}
        V_{eff}(\vr) = V(\vr) + \int d^3\vr' \frac{\rho(\vr)}{|\vr - \vr'|} + V_{xc}^{[\rho]}(\vr)
    \end{equation}

    $V_{xc}^{[\rho]}$ tu predstavlja izmenjalno-korelacijski potencial in ga precej natančno dobimo z aproksimacijo
    lokalne gostote (LDA).

    \section{Implementacija}

    \ldots

    \iffalse
    \begin{figure}
        \centering
        \includegraphics[width = \textwidth]{slika1.pdf}
        \caption{Energije verige preračunane na delec. Najnižja vrednost, do katere pridemo na grafu je $E_0/n \approx -1.767$}
        \label{slika1}
    \end{figure}

    \begin{figure}
        \centering
        \begin{subfigure}{\textwidth}
            \includegraphics[width = \textwidth]{slika3a.pdf}
        \end{subfigure}
        \begin{subfigure}{\textwidth}
            \includegraphics[width = \textwidth]{slika3b.pdf}
        \end{subfigure}
        \caption{Prikaza matrik spinskih korelacij v osnovnem stanju dveh različno dolgih verig.}
        \label{slika3}
    \end{figure}
    \fi

\end{document}
