\documentclass[a4paper]{article}
\usepackage[slovene]{babel}
\usepackage[utf8]{inputenc}
\usepackage[T1]{fontenc}
%\usepackage[margin=2cm, bottom=3cm, foot=1.5cm]{geometry}
\usepackage{float}
\usepackage{graphicx}
\usepackage{amsmath}
\usepackage{amssymb}
\usepackage{subcaption}
\usepackage{hyperref}
\usepackage{dirtytalk}

\newcommand{\tht}{\theta}
\newcommand{\Tht}{\Theta}
\newcommand{\dlt}{\delta}
\newcommand{\eps}{\epsilon}
\newcommand{\thalf}{\frac{3}{2}}
\newcommand{\ddx}[1]{\frac{d^2#1}{dx^2}}
\newcommand{\ddr}[2]{\frac{d^2#1}{d#2^2}}
\newcommand{\mddr}[3]{\frac{\partial^2#1}{\partial#2\partial#3}}

\newcommand{\der}[2]{\frac{d#1}{d#2}}
\newcommand{\pder}[2]{\frac{\partial#1}{\partial#2}}
\newcommand{\half}{\frac{1}{2}}
\newcommand{\forth}{\frac{1}{4}}
\newcommand{\q}{\underline{q}}
\newcommand{\p}{\underline{p}}
\newcommand{\x}{\underline{x}}
\newcommand{\liu}{\hat{\mathcal{L}}}
\newcommand{\bigO}[1]{\mathcal{O}\left( #1 \right)}
\newcommand{\pauli}{\mathbf{\sigma}}
\newcommand{\bra}[1]{\langle#1|}
\newcommand{\ket}[1]{|#1\rangle}
\newcommand{\id}[1]{\mathbf{1}_{2^{#1}}}
\newcommand{\tinv}{\frac{1}{\tau}}
\newcommand{\s}{\sigma}
\newcommand{\vs}{\vec{\s}}
\newcommand{\vr}{\vec{r}}
\newcommand{\vq}{\vec{q}}
\newcommand{\vv}{\vec{v}}
\newcommand{\vo}{\vec{\omega}}
\newcommand{\uvs}{\underline{\vs}}
\newcommand{\expected}[1]{\left\langle #1 \right\rangle}
\newcommand{\D}{\Delta}

\newcommand{\range}[2]{#1, \ldots, #2}
\newcommand{\seq}[2]{#1 \ldots #2}
\newcommand{\psiCoef}[2]{\psi_{\range{#1}{#2}}}
\newcommand{\psiCoeff}[3]{\psi_{#1, \range{#2}{#3}}}
\newcommand{\mpa}[2]{#1^{(#2)}_{s_#2}}
\newcommand{\us}{\underline{s}}
\newcommand{\up}{\uparrow}
\newcommand{\down}{\downarrow}

\begin{document}

    \title{\sc\large Višje računske metode\\
		\bigskip
		\bf\Large Problem elektronske strukture: metode povprečnega polja in DFT}
	\author{Mitja Vodnik, 28182041}
            \date{\today}
	\maketitle

    Kot preprost zgled uporabe DFT metode računamo elektronsko strukturo helijevega atoma v osnovnem stanju.
    Zanima nas torej elektronska gostota $\rho(\vr)$, ki po Hohenberg-Kohnovem izreku enolično določa osnovno
    stanje mnogodelčnega kvantno-mehanskega sistema.
    Ker ima helijev atom dva elektrona, je gostoto mogoče izraziti z le eno sferno simetrično orbitalo:

    \begin{equation}\label{eq1}
        \rho(\vr) = \rho(r) = 2|\phi(r)|^2
    \end{equation}

    To je orbitala, ki jo uporabimo v Kohn-Shamovem modelu za funkcional elektroske gostote.
    Tako dobimo DFT metodo, ki je ekvivalentna reševanju naslednjih Hartree-Fockovih enačb za orbitalo:

    \begin{equation}\label{eq2}
        \left( -\half \nabla^2 + V_{eff}(\vr) \right) \phi(\vr) = \epsilon \phi(\vr)
    \end{equation}

    \begin{equation}\label{eq3}
        V_{eff}(\vr) = V(\vr) + \int d^3\vr' \frac{\rho(\vr)}{|\vr - \vr'|} + V_{xc}^{[\rho]}(\vr)
    \end{equation}

    Prvs dva člena v izrazu za potencial~\ref{eq3} predstavljata elektrostatski potencial jedra in elektronov,
    $V_{xc}^{[\rho]}$ pa predstavlja izmenjalno-korelacijski potencial, ki ga precej natančno dobimo z aproksimacijo
    lokalne gostote (LDA).

    \section{Implementacija}

    Opišimo sedaj algoritem za numerično reševanje zgornjih Hartree-Fockovih enačb.
    Najprej zapišimo orbitalo kot:

    \begin{equation}\label{eq4}
        \phi(r) = \frac{u(r)}{r}
    \end{equation}

    Schr\"odingerjeva enačba~\ref{eq2} se tedaj prevede na:

    \begin{equation}\label{eq5}
        \left( -\half \ddr{}{r} + V_{eff}(r) \right) u(r) = \epsilon u(r), \quad u(0) = u(r_{max}) = 0
    \end{equation}

    Z $r_{max}$ smo označili valikost območja nakaterem rešujemo enačbe - biti mora dovolj veliko, da valovna funlcija
    $\phi$ do roba že pade na zanemarljivo vrednost.
    Efektivni potencial v zgornjem izrazu razpišemo kot:

    \begin{equation}\label{eq6}
        V_{eff} = -\frac{2}{r} + \frac{2U(r)}{r} + V_{xc}(r)
    \end{equation}

    Elektrostatski potencial elektronov smo zapisali kot $V_{el}(r) = \frac{2U(r)}{r}$ in ga dobimo iz Poissonove
    enačbe:

    \begin{equation}\label{eq7}
        \ddr{U(r)}{r} = - \frac{u^2(r)}{r}, \quad U(0) = 0, \quad U(r_{max}) = 1
    \end{equation}

    Približek za izmenjalno-korelacijski potencial dobimo po LDA metodi in se izrazi kot:

    \begin{equation}\label{eq8}
        V_{xc}(r) = - \left( \frac{3}{2\pi^2} \frac{u^2(r)}{r^2} \right)^{\frac{1}{3}}
    \end{equation}

    Algoritem za reševanje je naslednji:

    \begin{enumerate}
        \item Začnemo z orbitalo, ki ustreza osnovnemu stanju vodikovega atoma:

        \begin{equation}\label{eq9}
            u_0(r) = r\phi_0(r) \propto re^{-r}, \quad \int_0^{r_{max}} u_0^2(r)dr = 1
        \end{equation}

        \item Po enačbi~\ref{eq7} izračunamo elektrostatski potencial, po enačbi~\ref{eq8} pa
        izmenjalno-korelacijskega, in ju vstavimo v enačbo~\ref{eq6}, da dobimo efektivni potencial.

        \item Dobljeni potencial uporabimo v enačbi~\ref{eq5} da dobimo novo obliko orbitale $u(r)$.

        \item Točki 2 in 3 ponavljamo, dokler postopek ne konvergira.

    \end{enumerate}

    Kot rezultat algoritma dobimo elektronsko gostoto $\rho(r) = \frac{u^2(r)}{r^2}$ in ustrezno lastno energijo $E$,
    ki predstavlja zgolj zgornjo mejo za pravo energijo osnovnega stanja helija $E_0$.
    Lstna energija se izraža kot:

    \begin{equation}\label{eq10}
    E = 2\epsilon - \int_0^{r_{max}} \frac{2U(r)}{r} u^2(r) dr - \half \int_0^{r_{max}} V_{xc}(r) u^2(r) dr
    \end{equation}

    \subsection{Reševanje Schr\"odingerjeve enačbe}

    Enačbo~\ref{eq5} rešujemo s kombinacijo strelske metode in metode Numerova.
    Postopek je naslednji:

    \begin{enumerate}
        \item Izberemo dve začetni vrednosti parametra $\epsilon$, ki določita interval v katerem iščemo
        lastno vrednost enačbe.
        (Izberemo $\epsilon_1 = -1$ in $\epsilon_2 = 0$, ker vemo, da naj bi dobili $\epsilon = -0.52$.)
        \item Sedaj delamo bisekcijo po tem intervalu: vsakič za vrednost parametra na sredini intervala
        ($\epsilon = \frac{\epsilon_1 + \epsilon_2}{2}$) naredimo \say{strel} - z metodo Numerova rešimo
        enačbo~\ref{eq5} in preštejemo, koliko ničel ima dobljena rešitev.
        Iščemo osnovno stanje, torej rešitev, ki nima nobene ničle, razen na robovih.
        Če ima rešitev kakšno ničlo, je vrednost parametra previsoka, torej iskanje zožimo na spodnjo polovico intervala,
        če pa nima nobene ničle, se zožimo na zgornjo.
        Tako bomo skonvergirali k rešitvi, ki ničlo doseže ravno na robu intervala: $u(r_{max}) = 0$.
        (Za ničlo na začetku $u(r_{max})$ je poskrbleno v začetnem pogoju metode Numerova, ki je opisana v nadaljevanju.)
    \end{enumerate}

    Za metodo Numerova enačbo~\ref{eq5} malo preuredimo:

    \begin{equation}\label{eq11}
        u''(r) + 2\big(\epsilon - V(r)\big)u(r) = 0
    \end{equation}

    Sedaj prostor diskretiziramo na $N$ intervalov:

    \begin{equation}\label{eq12}
        h = \frac{r_{max}}{N}, \quad r_n = nh, \quad n = 1, \ldots, N
    \end{equation}

    \begin{equation}\label{eq13}
        V_n = V(r_n), \quad u_n = u(r_n)
    \end{equation}

    Metoda Numerova se tedaj glasi:

    \begin{equation}\label{eq14}
        u_{n+1} = \frac{(12 - 10f_n)u_n - f_{n-1}u_{n-1}}{f_{n+1}},
    \end{equation}

    kjer smo označili $f_n = \frac{h^2}{6} (\epsilon - V_n) + 1$.
    Potrebna sta še pogoja v prvih dveh točkah.
    Ker je pogoj v prvi kar ničelen, lahko v drugi točki izberemo poljubno vrednost, ki bo na koncu popravljena z
    normalizacijo.
    Izberemo torej:

    \begin{equation}\label{eq15}
        u_1 = 0, \quad u_2 = 1
    \end{equation}

    \subsection{Reševanje Poissonove enačbe}

    Tudi enačbo~\ref{eq7} rešujemo z metodo Numerova.
    Razlika pri tej je, da drugega začetnega pogoja ne moremo izbrati poljubno, saj potencial nima dodatne
    normalizacijske vezi, kot orbitala.
    Uporabimo postopek opisan v članku~\cite{it1}, ki nam omogoči izračun drugega pogoja brez izgube natančnosti.

    \section{Rezultati}

    DFT algoritem sem izvajal, dokler sprememba parametra $\epsilon$ po iteraciji ni padla pod $10^{-4}$.
    Za to je bilo potrebnih 13 iteracij.
    Dobljeni rezultat se ujema s predvidenim, če ga zaokrožimo na dve decimalni mesti:

    \begin{equation}\label{eq16}
        \epsilon = - 0.517, \quad E = - 2.724
    \end{equation}

    Kot rezultat dobimu tudi obliko orbitale in elektrostatski potencial elektronov v He atomu - ta rezultat je prikazan
    na sliki~\ref{slika1}.
    Za primerjavo je na sliki~\ref{slika2} dodan tudi analitičen rezltat za vodikov atom.
    Glavna razlika med obema atomoma je lepo ravidna na sliki~\ref{slika3}, kjer sta prikazani njuni elektronski
    gostoti - helijeva je nekoliko ožja od vodikove.
    Kvantitativno lahko to določimo tako, da izračunamo pričakovane vrednosti oddaljenosti od izhodišča:

    \begin{equation}\label{eq17}
        \expected{r} = \int_0^{r_{max}} r |\phi(r)|^2 r^2 dr
    \end{equation}

    Izvemo, da je radij helijevega atoma skoraj dvakrat manjši kot radij vodikovega:

    \begin{equation}\label{eq18}
        \expected{r}_{He} = 0.97, \quad \expected{r}_{H} = 1.5, \quad
        \frac{\expected{r}_{He}}{\expected{r}_{H}} = 0.65
    \end{equation}

    \begin{figure}
        \centering
        \includegraphics[width = \textwidth]{slika1.pdf}
        \caption{Orbitala in elektrostatski ter izmenjalno-interakcijski potencial helijevega atoma izračunan po
        DFT algoritmu.
        Diskretizacija je bila $N = 10^7$ pri $r_{max} = 15$; opravljenih je bilo 13 iteracij.}
        \label{slika1}
    \end{figure}

    \begin{figure}
        \centering
        \includegraphics[width = \textwidth]{slika2.pdf}
        \caption{Orbitala in elektrostatski potencial vodikovega atoma.}
        \label{slika2}
    \end{figure}

    \begin{figure}
        \centering
        \includegraphics[width = \textwidth]{slika3.pdf}
        \caption{Elektronski gostoti He in H atoma v logaritemski skali.}
        \label{slika3}
    \end{figure}

    \begin{thebibliography}{9}
        \bibitem{it1}Quiroz González, J. L. M., and D. Thompson.
            \say{Getting started with Numerov’s method.}
            Computers in Physics 11.5 (1997): 514-515.
    \end{thebibliography}

\end{document}
